\chapter{Storage and indexing}
I dati possono essere salvati su disco, in cui considerare pagine randomiche ha un costo fisso ma leggere pagine in sequenza \`e pi\`u efficente o tapes
in cui i dati sono recuperati solo sequenzialmente. Si chiama file organization un metodo di ordinare i file di records su memoria esterna. Un record id
\`e sufficiente per localizzarlo fisicamente e gli indici sono strutture dati che permettono di trovare i record id dati i valori nei campi di index search
key. Un buffer manager prende pagine dalla memoria esterna al main memory buffer pool. I layer di file e indici fanno chiamate al buffer manager. Esistono 
molte alternative, scelte in base alle necessit\`a:
\begin{itemize}
\item Heap file: ottima quando un tipico accesso \`e un file scan che recupera tutti i record.
\item Sorted files: ottima quando i record devono essere recuperati in un certo ordine o secondo un certo range. 
\item Indexes: strutture dati organizzate secondo alberi o hashing. 
\end{itemize}
\section{Indexes}
Un indice su un file rende pi\`u veloce le selezioni sul search key fields per l'indice. Ogni sottoinsieme di campi di una relazione pu\`o essere un search
key per l-indice della relazione. Non \`e una chiave della relazione. Un indice contiene una collezione di data entries e supporta un efficiente recupero di
tutte le data entries $k*$ con una key value $k$. Dato $k*$ si trovano record con la chiave $k$ in al pi\`u in un'operazione di disco I/O. 
\subsection{B+ Tree Indexes}
Le pagine foglia contengono data entries e sono concatenate, le pagine non foglia contengono unicamente index entries e sono utilizzate unicamente per 
direzionare la ricerca. Gli insert o delete richiedono di cercare dati foglia e poi cambiarne i valori. I cambi necessari possono riguardare anche
livelli pi\`u alti dell'albero. 
\subsection{Hash based indexes}
Sono buoni per le selezioni d'uguaglianza. Gli indici sono una collezione di buckets, formati da primary pages e un numero arbitrario di overflow pages. Si rende 
necessaria una hasing function $h$. Il risultato di $h(r)$ \`e il bucket cui $r$ appartiene $h$ controlla i search key fields di $r$. Per trovare il bucket di $r$ si 
prendono gli ultimi $global\ depht$ numeri di bit di $h(r)$. Per l'insert, se il bucket \`e pieno lo si deve splittare allocando una nuova page e 
redistribuendo. Se necessario si deve anche raddoppiare la directory per determinare il raddoppio si devono confrontare global depth e local depth. 
\subsection{Alternative di data entry $\mathbf{k*}$ nell'indice}
In una data entry $k*$ si possono salvare data record con valore di chiave $k$ o riduzioni di data records con valore di search key pari a $k$ o liste di riduzioni di data records con chiave di ricerva $k$. La
scelta per alternative data entries \`e ortogonale alla tecnica di indexing utilizzata per localizzare data entries con un dato valore di chiave $k$. Tipicamente gli indici contengono informazioni ausiliarie che
direzionano le ricerche verso le data entries desiderate.
\subsubsection{Salvare data records con valore chiave $\mathbf{k}$}
Se viene utilizzata questa alternativa la struttura di indice \`e un file di organizzazione per data records. Al massimo un indice di una data collezione di data records pu\`o utilizzare questa alternativa. Se vengono
salvati numerosi data records il numero di pagine che contiene data entrie \`e alto e implica che la dimensione delle informazioni ausiliare dell'indice \`e alta. 
\subsubsection{Salvare riduzioni di data records o liste di riduzioni}
Le data entries sono tipicamente molto minori rispetto a data records, pertanto funziona meglio con data records grandi, specialmente se le chiavi sono piccole. La terza alternativa \`e pi\`u compatta della 
seconda, ma porta a data entries di grandezza variabile anche se le chiavi di ricerca sono di lunghezza fissa. 
\section{Classificazione degli indici}
Se una search key contiene una primary key \`e chiamata un primary index, altrimenti un secondario. Se contiene una candidate key \`e detto unique. 
\subsection{Clustered e unclustered}
Se l'ordine dei data records \`e lo stesso o simile ordine delle data entries si chiama l'indice clustered, implicato dalla prima alternativa per le data entries. Un file pu\`o essere clustered su al pi\`u una search key.
Il costo di recupero dei dati attraverso indici dipende fortemente da questa caratteristica.
\subsubsection{Indici clustered}
Per creare un indice clustered su data entries salvate in un Heap file si deve ordinare l'heap file lasciando spazio libero nelle pages per inserts futuri e pagine di overflow potrebbero rendersi necessarie in futuro.
L'indice \`e clustered se ogni data entries con lo stesso valore di search key si trova sequenzialmente alle altre. 
\section{Scelta degli indici}
Per creare degli indici ottimi si considerino le queries pi\`u importanti e il piano migliore utilizzando gli indici correnti e si guardi se un pian migliore con indici addizionali, se \`e cos\`i lo si implementi. Prima id
creare un indice si deve considerare l'impatto sugli updates nella workload: si deve fare un trade-off in quanto gli indici velocizzano le queries ma rallentano le operazioni di update e richiedono spazio su disco.
\subsection{Linee guida sulla selezione degli indici}
\begin{itemize}
\item Gli attributi nelle clausole WHERE sono candidati per le chiavi di indice. Una condizione di match esatto suggerisce indici hash e query di range suggeriscono indici ad albero (il clustering aiuta molto in 
questo ambito). 
\item Search key multi attributo dovrebbero essere considerate quando una clausola WHERE contiene mutliple condizione. L'ordine degli attributi \`e importante per le range queries. Tali indici possono 
qualche volta rendere possibili strategie index-only.
\item Si provi a scegliere indici che diano benefici a pi\`u queries possibile. Essendo che un unico indice pu\`o essere clustrerizzato per relazione lo si scelga in base alle queries pi\`u importanti da cui ne 
trarrebbero beneficio.
\end{itemize}
\section{Indici con searh keys composte}
Una chiave di ricerca composta \`e una chiave di ricerca su una combinazione di campi. Le data entries nell'indice sono ordinate in modo da supportare range queries. 