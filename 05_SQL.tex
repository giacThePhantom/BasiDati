\chapter{SQL}
Una query basica in SQL ha la forma: \emph{SELECT [DISTINCT] target-list FROM relation-list WHERE qualification}, dove relation-list \`e una lista di nomi
di relazioni, possibilmente seguiti da una range-variable dopo ogni nome. La target-list \`e la lista di attributi delle relazioni in relation-list e 
qualification una comparazione tra attributi o tra attributo e costante con gli operatori $<$, $>$, $\le$, $\ge$, $=$, $\neq$. DISTINCT \`e una parola chiave
opzionale che indica se il risultato dovr\`a contenere duplicati. 
\section{Semantica}
La semantica di una query SQL \`e definita secondo questa strategia: si computa il prodotto cartesiano della relation-list, si eliminano le tuple che non 
rispettano la qualification e se DISTINCT \`e specificato si eliminano i duplicati.Le range-variables sono necessarie unicamente se la tabella compare due 
volte in FROM, ma \`e buona pratica usarle sempre. I campi nel risultato si possono rinominare in SELECT attraverso \emph{AS} o $=$, mentre \emph{LIKE} 
permette di effettuare dello string matching, dove \_ indica qualsiasi carattere e $\%$ per $0$ o pi\`u istanze di un carattere arbitrario. 
\section{Operazioni}
\subsection{Unione}
Pu\`o essere utilizzata per computare il risultato di ogni due insieme di tuple compatibili per l'unione attraverso la parola chiave \emph{UNION} o 
\emph{EXCEPT} che computa il compatibile.
\subsection{Intersezione}
Viene utilizzata per computare l'intersezione di tuple compatibili per l'unione attraverso la parola chiave \emph{INTERSECT}.
\section{Nested query}
Ogni clausola \emph{WHERE} pu\`o contenere a sua volta una query per fare paragoni con parole chiave come \emph{IN} o \emph{EXISTS} che verificano 
l'esistenza del primo elemento nella query successiva, \emph{UNIQUE} che verifica se \`e unico o operazione booleana con \emph{ANY, ALL, IN}. 
\subsection{Divisioni in SQL}
\emph{SELECT S.x FROM S WHERE NOT EXISTS (SELECT B.y FROM B WHERE NOT EXISTS (SELECT A.x FROM A WHERE A.y=B.y AND S.x=A.x))}, o 
\emph{SELECT S.x FROM S WHERE NOT EXISTS ((SELECT B.y FROM B EXCEPT (SELECT A.x FROM A WHERE S.x=B.x))}
\section{Operatori aggregati}
Sono un'estensione dall'algebra relazionale, permettono di svolgere operazioni su una singola colonna, sono: \emph{COUNT, AVG, MAX, MIN, SUM}.
\section{Grouping}
\emph{SELECT [DISTINCT] target-list FROM relation-list WHERE qualification GROUP BY grouping-list HAVING group-qualification}, dove la target list deve 
avere una lista attributi sottoinsieme della grouping-list. Ogni tupla di risposta corrisponde a un gruppo e gli attributi hanno un singolo valore per 
gruppo. Dopo aver eseguito il prodotto cartesiano della relation list si eliminano le tuple che non rispettano la qualificazione e quelle rimanenti sono
partizionate in gruppi secondo i valori degli attributi nella grouping list, successivamente \`e applicata la group-qualification per eliminare le tuple
che non la rispettano, tali espressioni devono avereun valore singolo per gruppo.
\section{Null Values}
Valori di campi nelle tuple possono essere sconosciuti o non specificati, pertanto per queste situazioni si introduce il valore \emph{null}. La presenza di 
questo valore rende molto difficile garantire la verit\`a delle condizioni \emph{WHERE}. Si rendono possibili o necessarie nuove operazioni come per esempio
gli outer joins. 
\section{Integrity constraints}
Gli IC descrivono delle condizioni che ogni istanza legale di una relazione deve soddisfare, pertanto inserts, updates, deletes che li violano non vengono 
considerati. Gli IC possono essere di dominio, di primary key, di foreign key e generali. Quelli di dominio sono sempre forzati.
\subsection{General constraints}
Utili quando constraints pi\`u geneali di quelli riguardanti unicamente le chiavi sono richiesti. Le queries possono essere utilizzate per creare questi
constraints e possono essere nominati attraverso \emph{Constraint $<$nome constraint$>$ CHECK $<$descrizione constraint$>$}.
\subsection{Constraints riguardanti pi\`u relazioni}
Se un constraint riguarda pi\`u relazioni viene chiamato asserzione e creato attraverso la query \emph{CREATE ASSERTION} in modo che non dipenda dall'effettiva popolazione di nessuna di quelle query. 
\section{Triggers}
I triggers sono procedure che cominciano in maniera automatica se un certo evento accade all'interno del DBMS, sono formati da tre parti: l'evento che
attiva il trigger, la condizione che verifica se l'azione del trigger debba essere applicata e l'azione che il trigger compie. 