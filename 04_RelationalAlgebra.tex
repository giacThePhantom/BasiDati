\chapter{Relational Algebra}
Un linguaggio di query permette di recuperare dati da un database. Il modello relazionale supporta linguaggi semplici ma potenti con forti fondamenti formali
basati sulla logica che permettono molte ottimizzazioni. Questi linguaggi non sono Touring completi e non sono utilizzati per calcoli complessi, ma 
gestiscono in maniera ottima accessi a grandi gruppi di dati. Due linguaggi di query matematici sono alla base delle implementazioni: la relational algebra
pi\`u operazionale, utile per rappresentare piani di esecuzione e il relational calculus che permette agli utenti di descrivere cosa vogliono e non come
computarlo e non \`e operazionale ma dichiarativo. Una query \`e applicata alle istanze delle relazioni. Gli schema delle relazioni di input e di output 
sono fissati, determinati dalla definizione dei costrutti del linguaggio di query. La notazione dei campi di una tabella pu\`o essere posizionale in modo 
da facilitare la formalizzazione o nominativa in modo da facilitare la lettura. 
\section{Operazioni}
\begin{itemize}
\item Selezione $\sigma$ seleziona un sottoinsieme di righe di una relazione.
\item Proiezione $\pi$ elimina colonne non volute dalla tabella.
\item Prodotto cartesiano $\times$ permette di combinare due relazioni.
\item Differenza di insiemi $-$ le tuple nella relazione $1$ ma non nella relazione $2$.
\item Unione $\cup$, tuple nella relazione $1$ e nella relazione $2$.
\end{itemize}
Essendo che ogni operazione ritorna una relazione possono essere composte. Questo dice che l'algebra \`e chiusa. 
\subsection{Proiezione}
Elimina tutti gli attributi che non esistono nella lista di proiezione, lo schema risultato \`e una tabella che contiene esattamente gli attributi elencati
con gli stessi nomi della tabela in input. Questa operazione deve eliminare i duplicati, cosa che non succede nei sistemi reali a meno che non venga 
specificato. 
\subsection{Selezione}
La selezione seleziona le righe che soddisfano la condizione di selezione. Non ci sono duplicati nel risultato e lo schema rimane identico allo schema della
tabella di input. 
\subsection{Unione, intersezione e differenza}
Queste operazioni prendono come input due tabelle compatibili rispetto all'unione: stesso numero di campi e campi corrispondenti dello stesso tipo. Lo 
schema di risultato \`e quello analogo alle due tabelle.
\subsection{Prodotto cartesiano}
Ogni riga della prima tabella \`e accoppiata con una della seconda, lo schema risultato ha tutti i campi delle tabelle con i nomi ereditati se possibile, 
altrimenti si rende necessaria un'operazione di rinominazione: $\rho(C(i\rightarrow name_1,\dots,  j\rightarrow name_n), R_1\times R_2)$.
\subsection{Joins}
\subsubsection{Condition join}
$R_1\bowtie_C R_2=\sigma_C(R_1\times R_2)$, lo schema di risultato \`e lo stesso del prodotto cartesiano ma contiene meno tuple. A volte chiamato theta-join.
\subsubsection{Equi-join}
Un caso speciale del conition join dove la condizione contiene solo uguaglianze. 
\subsubsection{Natural join}
Un equi-join su tutti i campi in comune. 
\subsection{Divisione}
Non supportata come operazione primitiva ma utile. Si definisce come $A/B=\{\langle x\rangle|\exists\langle x, y\rangle\in A \forall y\in B\}$, ovvero se 
l'insieme di $y$ valori associati con un valore $x$ in A contiene tutte i valori $y$ in $B$ allora $x\in A/B$. In generale $x$ e $y$ possono essere qualsiasi
lista di campi e $y$ \`e la lista di campi in $B$ e $x\cup y$ \`e la lista di campi in $A$. 
\subsubsection{Implementare una divisione}
Si computino tutte le tuple che non sono squalificate da qualche valore $y$ in $B$. Un valore $x$ \`e squalificato se unendoci un valore $y$ da $B$ si 
ottiene una tupla che non sta in $A$. Questi valori si ottengono attraverso $\pi_x((\pi_x(A)\times B)-A)$ e $A/B:\pi_x(A)-$ tuple squalificate.  
