\chapter{Introduzione}
Un database \`e una collezione di dati contenente entit\`a e relazioni riguardanti un'applicazione. Un database management system o DBMS \`e un software utilizzato per assistere nel mantenimento e utilizzo 
di larghe collezioni di dati.
\section{Gestire dati}
Esistono varie tipologie di DBMS ma ci si concentrer\`a sui relational database systems o RDBMSs che sono quelli pi\`u utilizzati.
\section{File systems e DBMS}
Il salvataggio di dati in un file system porta a problematiche di memoria in quanto la dimensione delle memorie principali \`e limitata, non \`e possibile indirizzare tutti i files. Si rende necessario scrivere 
programmi specifici per estrarre i dati dai files, proteggere i dati da cambi concorrenti che potrebbero creare inconsistenze. Si devono scrivere applicazioni ad-hoc in modo che dati possano essere ripristinati ad 
uno stato consistente se il sistema fallisce e creare un sistema per implementare politiche di sicurezza come l'user control. Un DBMS viene creato con lo scopo di ovviare a questi problemi.
\subsection{Vantaggi di un DBMS}
\begin{itemize}
\item Data indipendence: i programmi applicativi non sono dipendenti da dettagli della rappresentazione e salvataggio dei dati, pertanto il DBMS fornisce un'astrazione che li nasconde.
\item Efficient data access: un DBMS utilizza molte tecniche sofisticate per salvare e recuperare dati efficientemente.
\item Data integrity and security: se viene fatto l'accesso ai dati attraverso un DBMS questo pu\`o forzare constraints di integrit\`a e un access control che stabilisce che dati sono visibili a quali utenti.
\item Data administration: facilita il processo di gestione dei dati affidandolo a utenti esperti.
\item Concurrent access and crash recovery: un DBMS organizza gli accessi concorrenti ai dati in modo che un utente pu\`o considerare i dati come se fosse fatto l'accesso ad essi da un utente alla volta. Inoltre 
protegge gli utenti dagli effetti del fallimento del sistema.
\item Reduced application development time: fornendo importanti funzioni comuni a varie applicazioni unitamente all'interfaccia di alto livello permette la creazione pi\`u veloce e robusta di applicazioni.
\end{itemize}
Esistono ragioni per non utilizzare un DBMS: nel caso di applicazioni specifiche con constraints realtime o con poche ben definite operazioni critiche o nel caso un'applicazione debba gestire dati in modo non 
supportato dal linguaggio di query. 
\section{Descrivere e salvare dati in un DBMS}
Un data model \`e una collezione di descrizioni di dati di alto livello che nasconde molti dettagli. Un DBMS permette di descrivere i dati da salvare utilizzando tale modello. Il modello pi\`u utilizzato \`e il 
relational data model. Un'ulteriore astrazione rispetto al data model \`e il semantic data model ma un DBMS non \`e costruito introno a quello. Un esempio di semantic data model \`e l'entity reationship (ER) 
model. 
\subsection{Livelli di astrazione}
La descrizione di un database \`e creata da uno schema per ognuno dei tre livelli di astrazione: conceptual, physical e external. Un data definition language DLL \`e utilizzato per definire gli schemi esterni e 
concettuali come SQL. Le informazioni riguardanti gli schema sono salvati nei system catalogs.
\subsubsection{Conceptual schema}
Il conceptual schema o schema logico descrive i dati salvati nei termini del data model. 
\subsubsection{Physical schema}
Il physical schema specifica dettagli di salvataggio, ovvero riassume come il data model utilizzato \`e fisicamente salvato nella memoria. 
\subsubsection{External schema}
Gli external schema, solitamente espressi nel data model permettono di autorizzare e personalizzare gli accessi ai dati, pu\`o esistere un'external schema per ogni gruppo di utenti e consiste in una o pi\`u viste e 
relazioni dal conceptual schema. Le viste sono relazioni create a partire da specifiche degli utenti finali. 
\subsection{Data indipendence}
Si intende con data indipendence il fatto che i programmi applicativi sono isolati dai cambi nel modo in cui i dati sono strutturati e salvati, \`e creata dal conceptual e external schemas.  Esistono due livelli di data 
indipendence: logica rispetto ai cambi nelle relazioni e physical rispetto ai cambi della struttura di salvataggio dei dati su disco. 
\section{Query}
La facilit\`a con cui le informazioni possono essere ottenute da un database determina il suo valore. Per ricavare informazioni si utilizzano le queries, strutture create attraverso il linguaggio di query. Un 
esempio di linguaggio di query \`e il relational calculus basato su logica matematica. Un DBMS permette di creare, modificare e richiedere data attraverso un data manipulation language o DML che insieme al 
DDL forma il data sublanguage quando unito a un linguaggio host.
\section{Transaction management}
Per ordinare le richieste di vari utenti in modo da evitare inconsistenze da accessi concorrenti e proteggere gli utenti da fallimenti di sistema si devono gestire in maniera accurata le transazioni, ovvero ogni 
esecuzione di un programma utilizzatore in un DBMS. Transazioni parziali non sono ammesse. 
\subsection{Esecuzione concorrente di transazioni}
Un'importante funzione del DBMS \`e organizzare gli accessi ai dati in modo che ogni utente possa ignorare il fatto che altri possano star accedendo ai dati concorrentemente. Un locking protocol \`e un insieme 
di regole che ogni transazione deve seguire per permettere che nonostante diverse transazioni siano sovrapposte il loro risultato netto sia lo stesso che se fossero eseguite serialmente. Un lock \`e un 
meccanismo utilizzato per controllare l'accesso agli oggetti del DBMS, ne esistono di due tipi: shared locks su un oggetto possono essere mantenuti da due differenti transazioni allo stesso tempo, mentre un 
exclusive lock su un oggetto garantisce che nessun'altra transazione mantenga un lock sullo stesso oggetto. 
\subsection{Transazioni incomplete e crash di sistema}
Un DBMS deve garantire che cambi svolti da transazioni incomplete siano rimossi. Per farlo mantiene un log di tutte le scritture sul database: ogni azione di scrittura deve essere scritta su disco prima che il 
cambio corrispondente sia effettuato. Questa propriet\`a \`e chiamata Write-ahead log o WAL. Per far questo un DBMS deve poter spostare una pagina da memoria a disco. Il tempo di recupero da crash pu\`o 
essere ridotto forzando delle informazioni su disco periodicamente e creando un checkpoint.
\section{Struttura di un DBMS}
Un DBMS accetta comandi SQL generati da una variet\`a di interfacce utente, produce piani di query evaluation li esegue e ritorna la risposta.  Quando un utente presenta una query entra in gioco un query 
optimizer che usa le informazioni riguardo a come i dati sono salvati per produrre un piano di esecuzione efficiente. Un piano di esecuzione \`e un albero di operatori relazionali. Il codice che implementa questi 
operatori si trova sopra il livello di file e  dei metodi di accesso che supporta file che sono collezioni di pagine o collezioni di records. Gli heap files o file di pagine non ordinate e indici sono supportati. Questo 
livello si trova sopra il buffer manager che porta pagine da disco a memoria principale in necessit\`a di risposte per richieste di lettura. Il livello pi\`u basso \`e il disk space manager. Esistono un transaction 
manager, un lock manager e un recovery manager. 