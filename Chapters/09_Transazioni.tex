\chapter{Transazioni}
In applicazioni reali le operazioni possono essere richieste in maniera concorrente e due operazioni che riguardano le stesse tuple eseguite allo stesso tempo possono portare a risultati errati. Si definiscono 
pertanto le transazioni, un insieme di operazioni che devono essere svolte insieme. Due transazioni devono comportarsi come se fossero eseguite serialmente ma alcune loro parti possono sovrapporsi. Maggiore
la sovrapposizione, migliori le prestazioni del sistema. Una sequenza di operazioni \`e serializzabile se \`e uguale a una esecuzione seriale delle transazioni. Un ulteriore problema nasce quando due operazioni
devono per forza essere svolte atomicamente: se occorre un crash tra la prima e la seconda il database viene lasciato in uno stato incoerente. Pertanto si raccolgono operazioni sul database in transazioni, 
collezioni di diverse operazioni sul database che devono essere eseguite atomicamente: o si svolgono tutte o nessuna. L'SQL di default richiede che le transazioni siano eseguite in modo serializzabile: una 
transazione \`e eseguita nella sua interezza prima dell'altra. 
\section{Testare la serializzabilit\`a}
Sia una schedule un'esecuzione di transazioni con operazioni sovrapposte. Si determina un'invariante, una condizione che se vera prima dell'esecuzione delle transazioni deve rimanere vera anche al loro termine.
\subsection{Testing}
Per testare la serializzabilit\`a si devono ignorare i dettagli e testare se la schedule sar\`a serializzabile in ogni caso. Si analizzano solo la sequenza di letture e scritture. Il principio generale \`e di creare un
algoritmo che non dia falsi positivi. Falsi negativi non creano inconsistenze ma unicamente una degradazione delle prestazione. Si indichi:
\begin{itemize}
\item $r_T(X)$ la transazione $T$ che legge elementi dal database $X$.
\item $w_T(X)$ la transazione $T$ che scrive elementi del database $X$.
\item Se le transazioni sono indicizzate rispetto a $i$ si scrive $w_i(X)$ invece di $w_{T_i}(X)$.
\item Si indica come azione un'espressione nella forma $r_i(X)$ o $w_i(X)$. 
\item Una transazione $T_i$ \`e una sequenza di azioni indicizzate a $i$.
\item Una schedule $S$ \`e un insieme di transizioni $T$, \`e una sequenza di azioni in cui per ogni transazione $T_i$ in $T$ le azioni di $T_i$ appaiono in $S$ nello stesso ordine in cui appaiono nella definizione
di $T_i$.
\item Si dice che $S$ \`e sovrapposta sulle azioni delle transazioni che la compongono.
\end{itemize}
\subsection{Serializzabilit\`a del conflitto}
Data una schedule $S$ si esamino le azioni consecutive da differenti transazioni $T_i$ e $T_j$.
\begin{itemize}
\item $r_i(X)$ e $r_j(X)$ non sono mai in conflitto in quanto nessuna di esse cambia il valore degli elementi del database.
\item $r_i(X)$ e $w_j(Y)$ con $X\neq Y$ non sono in conflitto: se $T_j$ scrive $Y$ prima che $T_i$ legga $X$ il valore di $X$ non cambia. Anche nel caso contrario, la lettura di $X$ svolta da $T_i$ non ha effetti su $T_j$.
\item $w_i(X)$ e $r_j(Y)$, $X\neq Y$ non \`e in conflitto.
\item $w_i(X)$ e $w_j(Y)$, $X\neq Y$ non \`e in conflitto.
\end{itemize}
Se due azioni non sono in conflitto si possono invertire tra di loro senza cambiare il risultato.
\begin{itemize}
\item Due azioni della stessa transizione (ad esempio $r_i(X)$ e $w_i(Y)$) sono sempre in conflitto: non si pu\`o cambiare l'ordine di azioni in una transizione singola.
\item $w_i(X)$ e $w_j(X)$ sono in conflitto in quanto il valore di $X$ \`e quello computato da $T_j$. Invertendo l'ordine si ottiene $X$ con il valore computato da $T_i$.
\item $r_i(X)$ e $w_j(X)$ sono in conflitto: invertendo l'ordine il valore di $X$ letto da $T_i$ sar\`a quello scritto da $T_j$ che potrebbe essere differente dal valore precedente di $X$.
\item $w_i(X)$ e $r_j(X)$ sono in conflitto per ragioni simili.
\end{itemize}
Si nota come due azioni da transazioni diverse possono essere invertite a meno che involvono lo stesso elemento del database e almeno una \`e una scrittura. Data una schedule si fa il massimo numero possibile
di inversioni necessarie con lo scopo di trasformare una schedule in una seriale. Se si pu\`o fare la schedule originale \`e serializzabile. Due schedules sono conflict-equivalent se ognuna di esse pu\`o essere
trasformata nell'altra attraverso una sequenza di inversioni senza conflitto di azioni adiacenti. Una schedule si dice conflict-serializable se \`e conflict-equivalent a una schedule seriale. Conflict-serializability 
implica serializzabilit\`a. L'inverso \`e falso ma conflict-serializability \`e solitamente sufficiente. 
\subsection{Testare conflict-serializability}
Data una schedule $S$ che coinvolge due transazioni $T_1$ e $T_2$ $T_1$ ha precedenza su $T_2$ ($T_1<_S T_2$) se ci sono azioni $A_1$ di $T_1$ e $A_2$ di $T_2$ tali che:
\begin{itemize}
\item $A_1$ si trova prima di $A_2$ in $S$.
\item Sia $A_1$ che $A_2$ coinvolgono lo stesso elemento del database.
\item Almeno una di $A_1$ e $A_2$ \`e un'azione di scrittura.
\end{itemize}
Si noti come queste sono esattamente le condizioni per cui non si pu\`o invertire l'ordine di $A_1$ e $A_2$, pertanto $A_1$ appare prima di $A_2$ in ogni schedule che \`e conflict-equivalent ad $S$, pertanto 
ogni schedule seriale conflict-equivalent deve avere $T_1$ prima di $T_2$. 
\subsubsection{Grafo delle precedenze}
Si dice grafo delle precedenze un grafo orientato i cui nodi sono le transazioni di $S$ e un arco $(T_i, T_j)\in E$ se e solo se $T_i<_S T_j$. 
\subsubsection{Algoritmo}
Si costruisca il grafo delle precedenze, si provi se possiede cicli, se li possiede $S$ non \`e serializzabile, altrimenti qualsiasi ordine topologico dei nodi \`e un ordine seriale conflict-equivalent.
\subsubsection{Correttezza}
Si dimostri che se esiste un ciclo la schedule non \`e serializzabile: si supponga che esiste un ciclo di lunghezza $n$, allora le azioni di una transazione devono precedere quelle della transazione stessa e si
raggiunge pertanto una contraddizione. Si dimostra ora che se la schedule \`e serializzabile allora non esiste un ciclo: si provi per induzione sul numero di transizioni: se il grafo $n$ non ha cicli si possono 
riordinare le schedules actions utilizzando inversioni legali in modo che la schedule diventi seriale. Se $n=1$ la schedule deve essere seriale. Si assuma la tesi vera per tutti i valori minori di $n$ e lo si provi per 
$n$. La schedule consiste di azioni di transazioni $T_1, \dots, T_n$ con un grafo aciclico $S$, se il grafo \`e aciclico si deve avere un nodo $i$ senza archi in entrata, ovvero nessun nodo lo precede 
nell'ordinamento pertanto non ci sono altre azioni che coinvolgono una transazione $T_j$ tali che precedono delle azioni di $T_i$ e sono in conflitto con tali azioni. Si possono muovere tutte le azioni di $T_i$ 
all'inizio della schedule. 
\section{Aspetti pratici}
Non si pu\`o utilizzare l'algoritmo in quanto si dovrebbe aspettare che tutte le transazioni siano finite e si necessiterebbe di mantenere i dati per ogni transazione. L'algoritmo inoltre \`e inefficiente. Algoritmi
come timestamps permettono di controllare ogni transazione quando finisce e mantengono informazioni limitate su ogni transazione. Quando si scopre un problema si deve cancellare la transazione e disfare
ogni cambio fatto al database e portare questo cambio verso tutte le transazioni che nel frattempo hanno modificato il database. Questo funziona solo se questo tipo di problemi sono non frequenti, altrimenti
si fa in modo che le transazioni possano procedere unicamente se \`e possibile garantire la serializzabilit\`a. 
\section{Locks}
I locks sono mantenuti su elementi del database per evitare un comportamento non serializzabile. Intuitivamente una transazione che ottiene locks sull'elemento del database vi accede per impedire ad altre 
transazioni di accedervi allo stesso tempo. Questo garantisce la serializzabilit\`a. 
\subsection{Oggetti}
L'algoritmo accede ad oggetti che sono relazioni, tuple, parti di relazione, disk blocks. Pi\`u piccoli sono gli oggetti, pi\`u ne possono essere eseguiti in parallelo ma il costo di tenere conto di cosa viene lockato
pu\`o essere proibitivo. Si assume che gli oggetti siano tuple. Esiste uno scheduler che locka le tabelle. Tale scheduler riceve richieste dalle transazioni e permette loro di essere eseguite o le blocca fino a quando
\`e sicuro farlo. Idealmente lo scheduler inoltra una richiesta ogniqualvolta la sua esecuzione non pu\`o portare a uno stato inconsistente nel database. Un locking scheduler dovrebbe garantire conflict-serializability.
Ora le transazioni devono richiedere e rilasciare locks oltre che leggere e scrivere oggetti del database. Si devono soddisfare le propriet\`a di consistenza delle transazioni: una transazione pu\`o leggere o scrivere 
un elemento solo se precedentemente le \`e stato garantito un lock su quell'elemento e non \`e stato ancora rilasciato; se una transazione locka un elemento deve in un secondo momento rilasciarlo. 
La legalit\`a delle schedules: i locks devono avere il loro significato inteso: un lock dello stesso elemento pu\`o essere posseduto unicamente da una transazione. 
\subsubsection{Notazione}
\begin{itemize}
\item Sia $l_i(X)$: $T_i$ richiede un lock su un elemento $X$ del database. 
\item Si nota che \`e compito dello scheduler decidere se assecondare la richiesta o no.
\item $u_i(X)$: $T_i$ rilascia il suo lock su $X$. 
\item Consistenza: ogniqualvolta $T$ contiene un'azione $r_i(X)$ o $w_i(X)$ deve esserci un'azione precedente $l_i(X)$ e la prima occorrenza di  $u_i(X)$ deve trovarsi dopo l'azione.
\item Legalit\`a delle schedules: se una schedule contiene un'azione $l_i(X)$ seguita da $l_j(X)$ deve esiste un'azione $u_i(X)$ tra le due.   
\end{itemize}
\subsection{Locking scheduler}
Il compito del locking scheduler \`e di garantire richieste come $l_i(A)$ se e solo se la richiesta risulta in una schedule legale. Se la richiesta non \`e garantita la transizione \`e ritardata. La transazione aspetta 
fino a che lo scheduler garantisce la richiesta pi\`u tardi. Lo scheduler utilizza una tabella dei lock che dice per ogni elemento del database quale transazione o transizioni possiedono il lock per l'elemento.
\subsubsection{Two-phase locking}
Il two-phase locking (2PL): in ogni transizione tutte le azioni di lock precedono tutte le azioni di unlock in modo da garantire la serializzabilit\`a. Nella prima fase di una transazione i lock sono ottenuti, nella 
seconda fase sono rilasciati. Una transazione che rispetta le condizioni 2PL \`e chiamata una transazione two-phase-locked. 
\paragraph{Dimostrazione}
Si assuma che $S$ sia legale in una schedule 2PL. Si mostra come convertire $S$ in una conflict-equivalent schedule seriale. Si provi per induzione su $n$, il numero di transazioni in $S$. Si usano le operaizoni
di lock e unlock per la prova.
\begin{itemize}
\item $n=1$: $S$ \`e gi\`a una schedule seriale. 
\item Si assuma che il risultato valga per schedules con transazioni $n-1$: sia $T_i$ la transazione con la prima azione di unlock nella schedule, $u_i(X)$. Si mostra che si possono spostare tutte le azioni di $T_i$
all'inizio della schedule senza conflitti. Si consideri qualche azione $T_i$, $w_i(Y)$, affinch\`e possa essere preceduta in $S$ da qualche azione con conflitto come $w_j(Y)$ $j$ deve lockare l'item e $i$ deve
successivamente lockarla. Ma per definizione $u_i(X)$ \`e la prima operazione di unlock, pertanto deve essere prima di $u_j(Y)$, ma allora $u_i(X)$ appare prima di $l_i(Y)$ contrariamente al protocollo 2PL e 
si deve pertanto riscrivere $S$ come $T_i(T_{n-1}\cdots T_0)$.
\end{itemize}
Con questo algoritmo sono possibili deadlocks: tutte le transazioni stanno aspettando per la liberazione di un lock che non avverr\`a mai. 
