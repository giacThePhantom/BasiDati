\chapter{The entity relationship model}
Il modello relazioni entit\`a viene utilizzato per descrivere i dati coinvolti in un'operazione del mondo reale attraverso oggetti e le rispettive relazioni ed \`e utilizzato nella fase di database design concettuale. 
\section{Database design e ER diagram}
Il processo di database design pu\`o essere diviso in sei fasi, ma l'ER diagram viene utilizzato solo nelle prime tre:
\subsubsection{Requirement analysis}
In questa fase viene analizzato che tipo di dati vanno salvati nel database, che applicazione vada costruita al di sopra di esso e quali dati sono maggiormente soggetti a cambiamenti e pertanto influenti sulle 
performance. \`E solitamente un processo informale in cui si discute con gli utenti e in cui viene analizzato l'ambiente, analisi di documentazione di applicazioni da modificare o cambiare. 
\subsubsection{Conceptual database design}
Le informazioni raccolte nella fase precedente sono utilizzate per costruire un modello di alto livello dei dati che saranno salvati nel database con i constraints conosciuti per quanto riguarda i dati. In questa fase 
viene solitamente creato l'ER diagram in modo da crearne una semplice descrizione di come sia sviluppatori che utenti lo rappresentano. Permette sia una facile discussione con gli utenti che una 
rappresentazione abbastanza precisa da rendere la traduzione in un modello dati comprensibile al database utilizzato.
\subsubsection{Logical database design}
In questa fase viene scelto un DBMS per implementare il conceptual design, pertanto l'ER schema viene tradotto in un relational schema. Il risultato \`e un logical schema nel relational model. 
\subsubsection{Schema refinement}
In questa fase vengono analizzate le relazioni nello schema in modo da trovare eventuali problematiche e raffinarlo. Questa fase \`e guidata dalla teoria della normalizzazione delle relazioni.
\subsubsection{Physical database design}
In questa fase si considerano tipiche cariche di lavoro per il database e lo si raffina in modo che raggiunga i criteri di performance.
\subsubsection{Application and security design}
Si deve identificare entit\`a e processi coinvolti nell'applicazione, descrivere il loro ruolo nell'applicazione e garantire che i dati siano accessibili solo a chi ne abbia il privilegio. 
\section{Entit\`a, attributi e entity set}
Si definisce un'entit\`a un oggetto del mondo reale distinguibile dagli altri. \`E spesso utile identificare una collezione di simili entit\`a, chiamata entity set. Questi ultimi non possono essere divisi. Un entit\`a 
viene descritta da una serie di attributi e tutte le entit\`a nello stesso entity set hanno gli stessi attributi. La scelta degli attributi caratterizza il livello di dettaglio con cui si vuole rappresentare un entit\`a. Per ogni 
attributo si rende necessario identificare un dominio di possibili valori. Inoltre ad ogni entity set va associata una key, un set di attributi minimale che identifica univocamente l'entit\`a. Essendo che ci potrebbe 
essere pi\`u di una candidate key ne si designa una a primary key. Un entity set \`e rappresentato da un rettangolo mentre gli attributi da un ovale. La primary key \`e sottolineata. 
\section{Relazioni e relationship set}
Una relazione \`e un'associazione tra due o pi\`u entit\`a. Analogamente alle entit\`a pi\`u relazioni simili possono essere raccolte in relationship sets, che possono essere pensati come insiemi di ennuple: $
\{(e_1,\dots, e_n)|e_1\in E_1,\dots, e_n\in E_n\}$. Ogni ennupla denota una relazione che coinvolge $n$ entit\`a da $e_1$ a $e_n$, dove $e_i$ sta nell'entity set $E_n$. Una relazione pu\`o possedere degli 
attributi descrittivi in modo da registrare informazioni riguardanti la relazione ma non le entit\`a. Si definisce istanza di un relationship set un relationship set, ovvero come lo stato del relationship set in un 
qualche istante. Gli entity set che partecipano in una relazione non devono essere necessariamente distinti. Una relazione pu\`o infatti coinvolgere entit\`a appartenenti allo stesso entity set. In questo caso le 
due entit\`a svolgono diversi ruoli che deve essere identificato da due role identificator su ogni lato della relazione in modo da costituire un identificatore unico per la relazione. 
\section{Capacit\`a addizionali dell'ER model}
\subsection{Key constraints}
I key constraints si riferiscono alla numerazione delle relazioni, che si possono dividere in tre categorie principali:
\begin{itemize}
\item Many to many: le entit\`a di entrambi gli entity set si possono relazionare in qualsiasi numero con le entit\`a dell'altro entity set.
\item Many to one: le entit\`a di un entity set si possono relazionare con al pi\`u un'entit\`a dell'altro, mentre queste ultime in qualsiasi numero con le prime.
\item One to one: le entit\`a di entrambi gli entity set si possono relazionare con al pi\`u un'entit\`a dell'altro entity set.
\end{itemize}
\subsubsection{Key constraints per relazioni ternarie}
La convenzione sopra descritta pu\`o essere estesa a relazioni tra tre o pi\`u entit\`a: se un'entit\`a E possiede un key constraints nella relazione R ogni istanza di E appare al pi\`u una volta in ogni istanza della 
relazione R. 
\subsection{Participation constraints}
Indipendenti dai key constraints indicano quando la relazione sia totale o parziale. Nel primo caso ogni istanza dell'entit\`a deve apparire nell'istanza della relazione. 
\subsection{Weak entities}
Non \`e sempre vero che ogni entit\`a possieda una key. In questo caso si determinano delle weak entities, ovvero entit\`a considerate unicamente in relazione con l'entit\`a con cui sono in relazione. Si 
identificano con una combinazione tra loro attributi e la key della seconda entit\`a. Permettono di salvare dati dell'entit\`a unicamente quando \`e in relazione con un'altra chamata identifying owner. Per poter 
essere create si necessita che la owner entity e la weak entity siano in una one to many relationship set (ogni owner con una o pi\`u weak entities, ma ogni weak entity ha un singolo owner). Questo relationship 
set \`e detto identifiyng relationship set della weak entity. La weak entity deve avere una total partecipation con l'identifiyng relationship set. 
\subsection{Class hierarchy}
In alcuni casi risulta naturale classificare le entit\`a di un entity set in sottoclassi. In questo modo una sottoclasse eredita tutti gli attributi e quando richiesta la classe padre vengono restituite anche le classi figli. 
La class hierarchy pu\`o essere considerata in due modi:
\begin{itemize}
\item Ogni super classe \`e specializzata in sottoclassi. La specializzazione \`e il processo di identificare un sottoinsieme di un entity set (la classe padre) che condivide delle caratteristiche. Tipicamente \`e creata 
prima la superclasse e poi le sottoclassi con gli attributi specifici. 
\item Le sottoclassi sono generalizzate nella superclasse. In questo modus operandi vengono identificate delle caratteristiche comuni di una collezione di classi che vengono poi raccolte in una superclasse. 
\end{itemize}
Si possono specificare due tipologie di constraints per le gerarchie: di overlap determina se due sottoclassi possono contenere la stessa entit\`a, di covering determina se le entit\`a della sottoclasse includono 
tutte le entit\`a della superclasse. Esistono due motivi principali per identificare sottoclassi:
\begin{itemize}
\item Aggiungere attributi descrittivi che hanno senso solo per la sottoclasse.
\item Per identificare un entity set che partecipa in una relazione.
\end{itemize}
\subsection{Aggregation}
In alcuni casi si rende necessario modellare una relazione tra una collezione di entit\`a e relazioni. Per fare questo si usa l'aggregazione che ci permette di indicare che un relationship set partecipa in un altro 
relationship set in modo che il primo sia considerato come un'identit\`a. L'uso di un aggregazione invece di una relazione ternaria va usato in caso di attributi alle relazioni o a particolari constraints sulla 
relazione. 
\section{Conceptual design con ER model}
\subsection{Entit\`a o attributo}
Durante il processo di design pu\`o non essere chiaro se un attributo debba essere considerato come un'entit\`a a s\`e stante, questo va reso necessario quando:
\begin{itemize}
\item Vanno salvati pi\`u attributi per entit\`a.
\item La struttura dell'attributo \`e complessa e la si vuole catturare nel database.
\item Un attributo rappresenta un set di valori.
\end{itemize}
\subsection{Entit\`a o relazione}
Una relazione con troppi attributi porta a una perdita di efficienza di spazio, pertanto conviene trasformarla in alcuni casi in un'entit\`a.
\subsection{Binary or ternary relationship}
Una relazione ternaria deve essere trasformata da una serie di relazioni binarie quando un'entit\`a non pu\`o essere in relazione con pi\`u di un'altra, la stessa entit\`a deve essere sempre in relazione con l'altra, 
\`e una weak entity. Il viceversa \`e necessario quando le due relazioni sono strettamente correlate, ovvero se avviene una deve avvenire anche l'altra e non si pu\`o rappresentare un attributo di una delle 
relazioni con chiarezza.
\subsection{Relazione ternaria o aggragazione}
La scelta tra relazione ternaria o aggregazione \`e per principalmente determinata dall'esistenza di una relazione con un relationship set, ma anche da certi constraints che si vogliono imporre: come ad esempio 
un record da salvare della relazione con l'aggregazione o quando una relazione \`e many to one.