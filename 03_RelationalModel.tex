\chapter{Il modello relazionale}
Il modello relazionale considera il database come una collezione di una o pi\`u relazioni dove ogni relazione \`e una tabella. Una relazione consiste di uno relation schema e un relation instance: l'istanza \`e una 
tabella e lo schema descrive le teste di colonna per le tabelle. Lo schema specifica il nome della relazione, il nome di ogni campo o colonna o attributo e il dominio di ogni campo. Un dominio \`e riferito dal suo 
nome e possiede un iniseme di valori associati. Uno schema di una relazione \`e nella forma $<Table name>(att_1:\ dom_1, \dots, att_n:\ dom_n)$. Un'istanza di una relazione \`e un insieme di tuple dette records 
in cui ogni tupla ha lo stesso numero di campi dello schema relazionale.  L'istanza pu\`o essere pensata come la tabella in cui ogni tupla \`e una riga e tutte le righe hanno lo stesso numero di campi. Ogni 
relazione \`e definita da un insieme di tuple uniche, ovvero tutte diverse. I campi sono convenzionalmente numerati. Uno schema relazionale specifica il dominio di ogni attributo nell'istanza. Questi constraint di 
dominio specificano che i valori che appaiono nella colonna devono essere presi da quelli di tale dominio. Considerando lo schema superiore $\{<att_1:\ d_1, \dots, att_n:\ d_n>\|d_1\in Dom_1, \dots, d_n\in 
Dom_n\}$. Le istanze di relazioni sono instanze che soddisfano i constraints di dominio nel relational schema. Il grado o arity di una relazione \`e il numero di campi, mentre la sua caridnalit\`a \`e il numero di 
tuple in un'istanza. Un database relazionale \`e un insieme di relazioni di nome diverso. Un relational database schema \`e una collezione di schemi per le relazioni del database. Si dice istanza di un database 
relazionale una collezione di istanze di relazioni.
\section{Creare e modificare relazioni utilizzando SQL}
il linguaggio SQL utilizza la parola table per denotare relazioni. Il sottoinsieme di SQL per la creazione, eliminazione e modifica di tabelle \`e detto DDL. Per creare nuove tabelle si utilizza il comando 
\emph{CREATE TABLE $<$nome tabella$>$($att_1$  $dom_1$, $\dots$, $att_n$  $dom_n$)}. Le tuple sono inserite attraverso il comando \emph{INSERT INTO $<$caratterizzazione tabella$>$ VALUES 
$<$tupla$>$} su pu\`o omettere la lista dei campi in INTO, ma \`e buona pratica includerla in modo da chiarire univocamente l'ordine. Si possono eliminare tuple attraverso il comando \emph{DELETE FROM 
$<$caratterizzazione tabella$>$ WHERE $<$caratteristica delle tuple da eliminare$>$}. Si possono modificare i valori di una tupla esistente attraverso il comando \emph{UPDATE  $<$caratterizzazione 
tabella$>$ SET $<$elementi da modificare$>$ WHERE $<$caratteristica delle tuple da eliminare$>$}. La clausola WHERE \`e applicata prima di SET e se il valore della colonna \`e utilizzato per modificarlo 
nella parte destra dell'uguaglianza viene utilizzato il valore vecchio. 
\section{Integrity constraints sulle relazioni}
In integrity constaint \`e una condizione specificata su uno schema di database che restringe i dati che possono essere salvati su un'istanza del database. Se tale istanza li soddisfa tutti \`e detta legale. Un DBMS 
forza gli integrity constraints nel senso che permette di salvare nel database solo istanze legali. Gli integrity constraints sono specificati e forzati a tempi diversi: sono specificati durante la definizione dello 
schema, mentre quando un applicazione di database gira il DBMS controlla le violazioni e vieta cambi ai dati che violerebbero gli IC. In alcuni casi potrebbe compensare i cambiamenti ai dati in modo che si 
evitano violazioni. Un esempio di IC sono i domain constraints.
\subsection{Key constraints}
Un key constraint \`e una proposizione secondo cui un sottoinsieme minimo di campi di una relazione \`e un identificatore unico per la relazione. Questo insieme \`e detto candidate key per la relazione. Una 
candidate key pertanto determina che due tuple distinte in un'istanza legale non possono avere valori identici in tutti i campi della chiave e nessun sottoinsieme dell'insieme che forma la candidate key \`e un 
identificatore unico per la chiave. Un insieme che contenga oltre alla chiave ulteriori campi \`e detto superkey. Ogni relazione \`e garantita di avere una chiave e se non diversamente specificato tutti i campi la 
formano. Una relazione pu\`o avere diverse candidate key e di tutte queste si pu\`o identificare una primary key e intuituvamente una tupla pu\`o essere riferita da ogni parte del database salvando i valori dei 
campi della sua primary key. 
\subsubsection{Specificare key constraints in SQL}
In SQL per dichiarare che un sottoinsieme costituisce una chiave si utilizza l'UNIQUE constraint. Al pi\`u una di queste candidate keys pu\`o essere dichiarata come primary key attravero PRIMARY KEY constraint. 
\emph{CREATE TABLE $<$nome tabella$>$($att_1$  $dom_1$, $\dots$, $att_n$  $dom_n$, UNIQUE($<$candidate key$>$), CONSTRAINT $<$nome constraints$>$ PRIMARY KEY($<$attributi$>$))}. 
CONSTRAINT permette di nominare un constraint in modo che quando venga violato venga ritornato il nome cos\`i specificato.
\subsection{Foreign key constraints}
Pu\`o capitare che l'informazione salvata in una relazione sia collegata a informazioni legate in altre relazioni. Se una delle relazioni \`e modificata, l'altra deve essere controllata e cambiata per mantenere i dati 
consistenti. Un IC che riguarda entrambe le relazioni deve essere specificata in modo da permettere questi controlli. In questo caso si rende necessario un foreign key constraint. Un foreign key constraints
\`e un campo di una relazione che si riferisce ad un'altra tabella e deve essere identica per numero di campi e tipi ad essa. Una foreign key si pu\`o riferire alla stessa relazione. Si introduce il valore \textbf{null}
di un campo che vuol dire che il valore nel campo \`e sconosciuto o non applicabile. Sono possibili nelle foreign key ma non nelle primary key. 
\subsubsection{Specificare foreign key constraints in SQL}
\emph{CREATE TABLE $<$nome tabella$>$($att_1$  $dom_1$, $\dots$, $att_n$  $dom_n$, FOREIGN KEY ($<$nome campi$>$) REFERENCES $<$nome tabella$>$)} che obbliga ai campi della foreign
key di apparire uguali nella tabella referenziata. 
\section{Constraints generali}
Si pu\`o aumentare la granularit\`a dei constraints attraverso table constraints e assertions: le prime sono associate ad una tabella e controllate quando \`e modificata, mentre le assertions agiscono su pi\`u tabelle. 
\section{Forzare integrity constraints}
Se i constraints su primary key e UNIQUE sono di facile soluzione, SQL crea diverse soluzioni per gestire la violazione dei foreign key constraints: 
\begin{itemize}
\item Se \`e inserita una riga con una foreign key che non esiste nella tabella referenziata l'INSERT viene respinto.
\item Se una riga della tabella referenziata \`e eliminata o aggiornata si pu\`o:
\begin{itemize}
\item Eliminare tutte le righe della tabella che la referenzia con lo stesso valore. \emph{CASCADE}.
\item Non permettere l'eliminazione se una riga la referenzia. \emph{NO ACTION}.
\item Settare ogni valore che si riferiva a tale riga ad un valore di default. \emph{SET DEFAULT}, dove il valore di default \`e specificato nella definizione del campo.
\item Se non \`e parte della primary key della tabella referenziata si pu\`o settare a \emph{null}. \emph{SET NULL}
\end{itemize}
\end{itemize}
SQL permette in caso di eliminazione o aggiornamento di scegliere una delle quattro opzioni specificandole come parte della dichiarazione della foreign key attraverso le parole chiave scritte sopra precedute da
\emph{ON DELETE} o \emph{ON UPDATE}. 
\subsection{Transazioni e constraints}
Di default i constraints sono controllati alla fine di ogni SQL statement che potrebbe generare una violazione. Questo controllo pu\`o essere troppo rigido e pertanto si possono separare i constraints in 
\emph{DEFERRED} o \emph{IMMEDIATE}: il primo tipo viene controllato al tempo di commit.
\section{Querying dati relazionali}
Una relational database query \`e una domanda riguardante i dati e la risposta \`e una nuova relazione che li contiene. SQL \`e il linguaggio di query pi\`u utilizzato. Si possono visualizzare i dati attraverso il 
comando \emph{SELECT $<$nome campi $*$ per tutti$>$ FROM $<$nome tabella$>$ WHERE $<$condizioni su tabella$>$}. Si possono combinare informazioni presenti su tabelle diverse. 
\section{Trasformazioni da ER a relational}
Il modello ER \`e un modo conveniente per rappresentare un design di database di alto livello che poi viene tradotto nello schema relazionale per poi essere implementato.
\subsection{Da entity sets a tables}
Un entity set \`e mappato in una relazione in modo che ogni attributo di esso diventi un attributo della tabella. 
\subsection{Da relationship sets senza constraints a tables}
Un relationship set \`e mappato a una relazione. Si consideri un relationship set senza chiave e constraints di partecipazione. Per rappresentarlo si deve essere capaci di identificare ogni entit\`a partecipante
e dare valori agli attributi descrittivi, pertanto gli attributi di una relazione contengono: 
\begin{itemize}
\item Gli attributi primary key di ogni entity set partecipante come foreign key.
\item Gli attributi descrittivi dell relationship set.
\end{itemize}
Gli attributi non descrittivi formano una superkey.
\subsection{Da relationship sets con key constraints a tables}
Se un relationship set coinvolge $n$ entity set e qualche $m$ di essi sono collegati attraverso frecce la chiave di ciascuno di essi costituisce una chiave per la relazione dove il relationship set \`e mappato. Un 
ulteriore metodo \`e rappresentato da includere informazioni della relazione nelle tavole corrispondenti agli entity set con la freccia, in particolare se un relationship set coinvolge $n$ entity set e qualcuno di 
essi \`e linkato attraverso una freccia la relazione corrispondente a ognuno di questi set pu\`o essere aumentata includendo la foreign key delle altre e gli attributi della relazione. Queste foreign key possono 
avere null value e pertanto dello spazio pu\`o essere sprecato, risparmiando sulla difficolt\`a delle queries. 
\subsection{Da relationship set con partecipation constraints a tables}
Questo viene implementato attraverso le assertions: si deve garantire che una foreign key appaia con non null values. 
\subsection{Da weak entities a tables}
Un weak entity set partecipa sempre in una relaizone one-to-many relazione binaria e possiede un key constraint e total partetipation. Vengono pertanto tradotte utilizzando una foreign key della entity da cui
si riferiscono. Si deve comunque prendere in considerazione che possiede solo una chiave parziale e che quando un'entit\`a padrona viene eliminata deve essere eliminata anche la weak entity. La primary
key \`e pertanto una combinazione della foreign key e un altro attributo e l'eliminazione deve avvenire \emph{ON DELETE CASCADE}.
\subsection{Tradurre gerarchie di classi}
Esistono due approcci per generare gerarchie di classi: 
\begin{itemize}
\item Mappare ogni elemento della classe padre nella classe figlio attraverso la foreign key della superclasse che svolge il ruolo di chiave primaria e per referenziare tutti gli attributi del padre. Un eliminazione
del padre deve essere fatto a cascata sul figlio.
\item Creare una relazione che possiede i propri attributi e tutti gli attributi della classe padre.
\end{itemize}
Il primo approccio \`e generalmente applicabile. Il secondo unicamente se la superclasse non contiene elementi che non sono presenti nelle classi figlie in quanto non esiste.
\subsection{Tradurre aggregazioni}
Quando si vuole fare una relazione con un'aggregazione si crea una relazione con la chiave dell'altro elemento, la chiave della relazione dell'aggregazione e gli attributi descrittivi. Se la relazione centrale 
dell'aggregazione non possiede attributi pu\`o essere ottenuta attraverso la coppia delle chiavi degli entity set nell'aggregazione come foreign key. 
\section{Viste}
Si dice vista una tabella le cui righe non sono salvate esplicitamente nel database ma sono computate quando rese necessarie da una view definition. Si creano come \emph{CREATE VIEW} con un select.
Possono essere utilizzate come una tabella base. Quando vengono utilizzate la view definition viene prima computata e poi portata come valore d'ingresso.
\subsection{Data indipendence, security}
Il meccanismo delle viste permette il supporto per la data indipendece logica nel relational model: pu\`o essere utilizzato per definire relazioni che mascherano cambi nello schema concettuale. Inoltre 
permettono di definire viste che danno a gruppi di utenti solo le informazioni che hanno il privilegio di vedere. 
\subsection{Aggiornamenti su viste}
Si permette di aggiornare viste solo se sono definite su una singola tabella base. La vista \`e chiamata updatable view. 
\section{Eliminare e alterare tabelle e viste}
Tabelle e fiste possono essere rimosse attraverso \emph{DROP TABLE} o alterate attraverso \emph{ALTER TABLE}.
